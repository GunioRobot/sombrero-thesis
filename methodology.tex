%\chapter{Project Progression}
\section{Methodology}
\subsection{Overview}

As we started to plan our project sombrero, we decided to do it in a fresh, modern and agile way, because after 4 years of programming in only imperative and boilerplate stuffed languages with project planning methods like RUP, we thought it would be nice to try something different. So we chose Scala as a lightweight and mostly boilerplate free language for our server side programming, but without a framework this would have been too much work. Luckily there is a web framework in Scala and it's called Lift. It borrows its ideas and concepts from the other most popular frameworks out there. Probably the most popular one is Ruby on Rails. It's created by 37 signals and used by many well known companies like Xing and Twitter.

Because of the methodology's popularity 37 signals wrote a book on how to use it properly in a project, but to make it more adaptable for different situations it was written very abstract. So you could apply it for almost any project. It's called Getting Real. The agile manifest was used as a model to create it. To use this knowledge Gabriel wrote a short adaption of Getting Real for our project. So with this kind of methodology to lead our way we were able to organize our project a fresh and agile way.

  \clearpage
\subsection{Agile Manifesto}
 ``The modern definition of agile software development evolved in the 1990s as part of a reaction against "heavyweight" methods, perceived to be typified by a heavily regulated, regimented, micro-managed use of the waterfall model of development. The processes originating from this use of the waterfall model were seen as bureaucratic, slow, demeaning, and inconsistent with the ways in which software developers actually perform effective work.''\cite{wikipedia.org:agile}

 The Agile Manifesto is published in 2001.

 ``The Principals are:
  \begin{itemize}
    \item \textbf{Individuals and interactions} over processes and tools
    \item \textbf{Working software} over comprehensive documentation
    \item \textbf{Customer collaboration} over contract negotiation
    \item \textbf{Responding to change} over following a plan''\cite{agilemanifesto.org}
  \end{itemize}

\subsection{Getting Real}
    ``Getting Real is a smaller, faster, better way to build software. It is about iterations and lowering the cost of change. Getting Real is all about launching, tweaking, and constantly improving which makes it a perfect approach for web-based software.''\cite{37signals:10}

\subsection{Project Manifest}
    The length of an iteration was 4 weeks, holidays not included. Before the start of an iteration a backlog had to be written and posted to the Project Blog. Communication was done mostly per e-Mail. In special cases, like the end of an iteration, a meeting had to be held. The result of every iteration had to be a functional build. Every work related to sombrero had to be documented on the respective Google Docs\footnote[1]{Google Docs is a free, Web-based word processor offered by Google.} spreadsheet.

    Everything else concerning project management had to be done conforming to the rules of the Agile Manifest and the adoption of Getting Real for sombrero.

\subsection{Google Sites, Docs and Groups}
    As a project management tool we used Google Sites\footnote[2]{Google Sites is a like a wiki with Google Gadgets and Blog.} and Google Docs. On Google Sites we had project blog were we posted the specification of every iteration before the beginning of it. After that we posted the allocation of responsibilities. After the end of each iteration we published the result.

    We used Google Docs as our project log. Both of us had a spreadsheet and in each row we filled the respective date, the work we did and the hours spent in.
\subsection{Pivotaltracker}
    The time management during the documentation work was done by Pivotaltracker, a web based tool that our project supervisor Prof. Harald Haberstroh showed us. We really liked working with it and its structure fits our needs much better than Google Sites.