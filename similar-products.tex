\section{Similar Products}

Prior to developing sombrero, we looked at other open source KNX management tools in order to collect ideas and learn from their mistakes. However, we did not find a lot of software that even had KNX support to start with.


\subsection{KNX@home}

KNX@home is a KNX control software similar to sombrero. It is being developed by the University of Applied Sciences in Deggendorf, Germany. The sombrero project originated when Gabriel tried to set up KNX@home during his summer internship. Despite being written in Java and advertised as platform independent, it has lots of dependencies and a very complicated architecture. The only way to get it running was to use the offered live CD. It was then that Gabriel noticed they were only supporting Lamps and had most likely spent most of their time getting their architecture and JavaScript frontend to work. As a consequence, we decided to use a straightforward architecture and powerful frameworks so we could concentrate on more important issues.


\subsection{LinuxMCE}

The goal of LinuxMCE is not to create a KNX control software, but rather a whole smart home framework, including media, telecom, security and building automation. As such, the focus is not on proper KNX support, and setting it up is difficult and requires advanced knowledge of Linux. Also, as the name suggests, LinuxMCE is tied to the Linux platform.
