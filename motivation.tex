%\chapter{Introduction}
\section{Motivation}
The motivation behind sombrero were the new technologies we got to try and use to create it. After 3 years of developing with the popular main stream languages, we had enough of all this boilerplate. So we looked around for a new way of writing code efficiently and we found Scala, a functional and object-oriented language. Both of us never programmed in a functional language before, that's why it was a great experience to learn about new programming techniques. Because of our interest in Scala, we went to the "Scala Days 2010" of our own accord, the fist Scala convention, to expand our horizons towards Scala.


Nowadays a great language isn't enough to build rich and boilerplate free applications. So we looked around for a framework in Scala and we found Lift. This framework borrows ideas from all the other great and popular server side frameworks out there and tries to combine their best features. The one that probably helped us the most was the powerful snippet feature, which allowed us to clearly separate our logic from the view part of the application.

Another new technology we got to learn was JQuery, a JavaScript framework. Almost every line of JavaScript code in sombrero is using the JQuery library. It's great to work with such a lightweight and easy to use framework and many popular users, like Google and Microsoft, prove us right. JQuery has also got also an UI framework, which consist of a CSS and JavaScript framework. JQuery UI allows you to combine view and logic to create widgets. All of them are themeable with the in-built CSS framework. It even provides a modular class-like abstraction for widgets, which we used to design our widgets and make them as reusable as possible.

Now we have got the server and the client logic, but what's with the design and layout? For this task we used simple CSS and YAML, a multicolumn layout CSS framework.

The  methodology we used to combine this rich toolset to a working application was Getting Real by 37 signals. It's oriented on agile software development and was written to teach small teams how to efficiently create great web applications with a small team of well chosen developers. Not everything in it can be applied to our project, but Gabriel Grill wrote an adaption that fit our needs.

We were very interested in doing our work in the promising field of home automation, because its influence in modern society is steadily increasing. Another big benefit was that we could be part in a project that could help to support environment friendly technologies and practices. KNX is an internationally standardized and widely used building automation system technology in the European area. The access to KNX in sombrero is managed through the Java API Calimero, which was developed by our client. Calimero provided a great and easy to use abstraction for accessing the devices in a house. 